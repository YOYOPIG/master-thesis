\documentclass{PHlab-thesis}

\addbibresource{thesis.bib}

\newcommand*\Department中文{資訊工程學研究所}
\newcommand*\Department英文{Institute of Computer Science and Information Engineering}

\newcommand*\ThesisTitle中文{EAGLE-GPU:使用圖形處理單元加速替代基因組可能性評估}
\newcommand*\ThesisTitle英文{EAGLE-GPU: Acceleration of Alternative Genome Likelihood Evaluation using Graphics Processing Unit}
%\newcommand*\ThesisNote中文{示例:其實徐翡曼是東京大學畢業的博士}% For real thesis omit, or use {初稿} etc.
%\newcommand*\ThesisNote英文{Just an example.  Fei-Man actually graduated from Tokyo Univ.}% For real thesis omit, or use {draft} etc.

\newcommand*\Student中文{盧宥霖}
\newcommand*\Student英文{You-Lin Lu}

\newcommand*\Advisor中文{賀保羅}
\newcommand*\Advisor英文{Paul Horton}

%% 果有共同指導老師可以用:
%% \newcommand*\CoAdvisorA中文{}
%% \newcommand*\CoAdvisorA英文{}
%% \newcommand*\CoAdvisorB中文{}
%% \newcommand*\CoAdvisorB英文{}


\newcommand*\YearMonth英文{July, 2022}
\newcommand*\YearMonth中文{111年7月}


\begin{document}


\newcommand*\Keywords英文{bioinformatics, genomics, string algorithms}
\newcommand*\Abstract英文{%TODO
We introduce MethylSeqLogo, an extension of sequence logos to vizualize DNA methylation.
}


\newcommand*\Keywords中文{生命科學、基因組、字串演算法}
\newcommand*\Abstract中文{%TODO
MethylSeqLogo...衍伸sequence logo的視覺化方法改善包括DNA甲基化的資訊。
}

\newcommand*\Acknowledgements{%
Thank you Prof. Horton%
}



\newcommand*\SelectFontsize[2]{\fontsize{#1}{#1}\selectfont\mdseries#2\par}
\newcommand*\SelectFontsizeBF[2]{\fontsize{#1}{#1}\selectfont\bfseries#2\par}
\newcommand*\SignatureRule[1][6]{\rule{#1cm}{0.3mm}}
\newcommand*\AddToContents[1]{\newpage\phantomsection\addcontentsline{toc}{chapter}{#1}}

\doublespace
\pagenumbering{gobble}
\renewcommand{\thefootnote}{\fnsymbol{footnote}}


\begin{center}
\vspace{2cm}
\SelectFontsizeBF{24}{%
\University中文\Department中文\\
\學位 論文}

\vfill
\SelectFontsizeBF{24}{\ThesisTitle中文}
\ifdefined\ThesisNote中文
\SelectFontsize{22}{\textit{\ThesisNote中文}}
\fi

\vspace{5mm}
\SelectFontsizeBF{22}{\ThesisTitle英文}
\ifdefined\ThesisNote英文
\SelectFontsize{20}{\textit{\ThesisNote英文}}
\fi

\vfill

\begin{minipage}{\linewidth}
{\setlength\tabcolsep{0pt}
%
\begin{tabular}{ Wr{5em} Wl{6em} Wr{5em} wl{7em} }
研究生:   & ~~\Student中文  &      Student: & ~~\Student英文\\
指導老師: & ~~\Advisor中文  &      Advisor: & ~~\Advisor英文\\
\ifdefined\CoAdvisorA中文
共同指導: & ~~\CoAdvisorA中文 &   Co-Advisor: & ~~\CoAdvisorA英文\\
\fi
\ifdefined\CoAdvisorB中文
         & ~~\CoAdvisorB中文 &   Co-Advisor: & ~~\CoAdvisorB英文\\
\fi
\end{tabular}
}
\end{minipage}

\vfill
\SelectFontsize{18}{%
National Cheng Kung University,\\
Tainan, Taiwan, R.O.C.\\
Thesis for \ifdef\PhD{Doctor of Philosophy}{Master of Science} Degree\\
\YearMonth英文}

\vfill
\SelectFontsize{20}{中華民國\YearMonth中文}
\end{center}



\ifdefined\optCommittee
\newpage
\begin{center}
\vspace{1cm}
\SelectFontsizeBF{24}{%
\University中文\Department中文\\
\學位 論文}
\vfill
\SelectFontsizeBF{20}{\ThesisTitle中文}
\end{center}

\vfill
\SelectFontsize{20}{%
\noindent 研究生:\Student中文\\
本論文業經審查及口試合格特此證明}


\begin{center}
\SelectFontsize{18pt}{論文考試委員}
\vfill
\SignatureRule \hspace*{1cm} \SignatureRule
\vfill

\SignatureRule \hspace*{1cm} \SignatureRule
\vfill

指導教授:\SignatureRule[8]
\vfill
  所長:\SignatureRule[8]

\vfill
\SelectFontsize{18}{中華民國 \hspace{2em} 年 \hspace{2em} 月 \hspace{2em} 日}
\end{center}


\newpage
\begin{center}
\vspace{1cm}
\SelectFontsize{18}{\University英文, \Department英文}
\SelectFontsize{19}{\ifdef\PhD{Ph.D.}{Master's} Degree Thesis}

\vfill
\SelectFontsizeBF{20}{\ThesisTitle英文}
\end{center}

\vfill
\SelectFontsize{18}{Student: \Student英文}

\SelectFontsize{18}{%
A thesis submitted to the graduate division in partial fulfillment of the requirement for the degree of
\ifdef\PhD{Doctor of \mbox{Philosophy}}{Master of Science}.
}

\vfill
\begin{center}
\SelectFontsize{18}{Approved by}

\vfill
\SignatureRule \hspace*{1cm} \SignatureRule

\vfill
\SignatureRule \hspace*{1cm} \SignatureRule

\vfill
Advisor: \SignatureRule[8]

\vfill
Chairman: \SignatureRule[8]

\vfill
\SelectFontsize{18}{\YearMonth英文}
\vspace*{20pt}
\end{center}
\fi% optCommittee


\AddToContents{中文摘要}
\setcounter{page}{1}
\pagenumbering{roman}


\begin{center}
\SelectFontsizeBF{24}{\ThesisTitle中文}

\vspace{4mm}
\SelectFontsize{18}{\Student中文\footnote[1]{學生} ~ \Advisor中文\footnote[2]{指導教授}}

\vspace{5mm}
\SelectFontsize{20}{國立成功大學\Department中文}

\vspace{12mm}
\makebox[2.7cm][c]{\SelectFontsizeBF{22}{摘要}}

\vspace{4mm}
\SelectFontsize{16}{\Abstract中文}

\vspace{4mm}
\begin{flushleft}
\SelectFontsize{16}{\textbf{關鍵詞:} \Keywords中文}
\end{flushleft}
\end{center}


\AddToContents{Abstract}
\begin{center}
\SelectFontsizeBF{22}{\ThesisTitle英文}

\vspace{4mm}
\SelectFontsize{18}{\Student英文\footnote[1]{Student} ~ \Advisor英文\footnote[2]{Advisor}}

\vspace{4mm}
\SelectFontsize{16}{\Department英文, National Cheng Kung University}

\vspace{12mm}
\SelectFontsizeBF{20}{Abstract}

\vspace{4mm}
\SelectFontsize{14}{\Abstract英文}
\end{center}

\vspace{4mm}
\begin{flushleft}
\SelectFontsize{16}{\textbf{Keywords:} \Keywords英文}
\end{flushleft}



\AddToContents{誌謝}
\begin{center}\SelectFontsizeBF{24}{誌謝}\end{center}

\vspace{4mm}
\Acknowledgements



\renewcommand{\contentsname}{CONTENTS}
\AddToContents{Contents}
\tableofcontents


\AddToContents{List of Tables}
\listoftables


\AddToContents{List of Figures}
\listoffigures
% 封面頁, 口委中英文簽名單, 誌謝, 中英文摘要, 論文目錄, 圖表目錄


%────────────────────  List of Symbols  ────────────────────
\renewcommand\nomgroup[1]{%
  \item[\bfseries
  \ifstrequal{#1}{A}{General}{%
  \ifstrequal{#1}{Z}{Gene/Protein Names}%
  }]}

\nomenclature[A]{$\lg$}{Logarithm base 2}
\nomenclature[A]{KL\ Divergence}{Kullback-Liebler Divergence}
\nomenclature[Z]{Myc}{MYC proto-oncogene}
\nomenclature[Z]{USF-1}{Upstream stimulatory factor 1}

\printnomenclature[5cm]

\newpage
\setcounter{page}{1}
\pagenumbering{arabic}



\chapter{Introduction}
Next generation sequencing has played a significant role in bioinformatics research for the past decade, where variant calling remains to be a challenging subtask. Different methods have been proposed for this specific task, including the successor of this study, EAGLE: Explicit Alternative Genome Likelihood Evaluator~\cite{kuo2018eagle}, published in 2017 by Tony Kuo et al. 
However, we noticed that the execution of EAGLE could take up to a decent amount of time, despite the usage of the provided multithreading option. Although the total execution time seems rather acceptible currently, we would like to further investigate potential oppurtunities of acceleration, regarding the upcoming third generation sequencing, where sequencing read lengths could easily grow up to 10,000 base pairs.
Since sequence alignment is a highly parrallel task, it was already shown in the original EAGLE research that multithreading could effectively speedup the whole process. In order to further accelerate the program, adding more threads would be a straightforward approach. While the total amount of threads that can be ran concurrently is constrained by the hardware, it is often capped around tens to twenties for modern mainstream CPUs. This is far from enough in comparison to human genome sequences, motivating us to search for other sources of acceleration, including the usage of additional devices. 
This is when Graphics Processing Unit(GPU), comes to our mind. Originally designed for graphics rendering, GPUs come with a lot more cores in comparison to CPUs, which is highly beneficial for data-parallel problems~\cite{navarro2014survey}. In the recent years, GPUs have been shown to greatly accelerate the process of a variety of tasks, including computer vision, deep learning, etc. Thus, we would like to investigate whether sequence alignment and variant calling, composed of several subtasks which are also highly data-parallel, can be can be accelerated with the aid of GPUs.
Here, we present a GPU accelerated version of EAGLE. Most of the probablistic model would be rather similar to the original method, while several adjustments were made to reduce memory copy between device and host, making it a better fit to the device-host parellel computing architecture. Experiments were made by using both real data and simulated data, concluding that ...

\chapter{Related Works}
In this chapter, we review recent works related in brief.
\section{EAGLE}
\section{NVBIO}
\section{GASAL2}
In this research, 

\chapter{Method}
\section{Proposed Scheme}
In the original EAGLE publication, whether a hypothesis fits the sequenced read data is determined by its likelihood, often measured by its ratio against the default hypothesis: the read being sequenced from the reference genome. In order to compute the likelihood for all possible read alignments, the probabilities of all read and genome segment pair are calculated and then further summed up in logarithm. Although this is undoubtly a straightforward solution, numerous time consuming operations were done in loop. Here, we unroll the nested loops and take advantage of the high parallel graphics processing units. 
Before making any adjustments to the program, we would like to first take a closer look at the probablistic model proposed in EAGLE.

% Should I put the thorough probablistic model here?

First of all, for any given pair of genome segment g and sequence read r, the likelihood of the read being sequenced from g is computed by product of the likelihood of all base pairs. Since the likelihood of each bp match is independant to other bp matches, they can be computed concurrently. This provides us the oppurtunity to accelerate this process by computing them in parallel, where the maximum number of operations that can be done in parallel is bounded by the length of the read r. Considering NGS read data, which is often around 100 to 150 base pairs in terms of read length, we can easily load the entire read onto the GPU and allocate a thread for each base, since the size is rather small in comparison to the memory capacity and the core counts of modern GPUs.

Next, we would like to further extend the degree of parallelism, seeking to utilize the performance provided by the GPU. Here, we take a closer look at the intermediate result, p(r|G). In most cases, knowing the exact genome segment g where read sequence r is sequenced from in advance is unrealistic. Thus, in the original method, for all possible genome segment g sampled from the hypothesis genome G, with the exact length with the given read r, were considered. P[r|g] are calculated iteratively, with the summation of their results being the final score of the given read. Since the total amount of segments to be considered is related to the length of read sequence(=2*lr), the execution time required grows in polynomial time with the sequence length. Apparently, the results of P[r|g] for different genome segment g is also independant, opening an oppurtunity for parallelism. In addition to the entire read r, we also loaded the selected genome G to the GPU, allocating a block of threads for each possible genome segment g. Within each block, we execute the same instructions as described in the previous section, calculating the probabilities for each site in their corresponding thread.

Through the above method, we have already reduced the amount of time required to execute the program. This also allows us to revisit a term that was previously ignored: insertion/deletion errors in read sequences. Since indel errors not only modifies the sequence but also introduce shifts, it would increase the time complexity, proportional to the length of reads. With the previous implementation, where the operations were sequentially executed, this would sure intensify the computational cost and lead to longer execution time. Mainly considering generation sequencing data, which has a rather low probability of indel errors in the sequencing phase, an assumption that there was no indel sequencing errors was made, in order to avoid the problem descibed above. However, with the aid of GPUs, such operations can be done concurrently, reducing the amount of time required to compute the final result. In GPU-EAGLE implementation, a threshold value is input by the user, indicating the maximum number of indel errors to be taken into account. After all, the probability of such errors is still fairly low, making it pointless to compute all of the combinations of shifted reads and indels. Knowing the threshold beforehand, all of the possible reads having less indel errors than it is then listed out and assigned to the GPU. Similar to the base case where no indel errors was considered, we would need a block of threads for each pair of read r and genome segment g. With regard to the same read having multiple variations due to the introduced indel errors, the GPU would launch a two dimensional block instead of the original one dimensional block, with the extra dimension representing the index of read in the list of possible reads with indel. To allow such indeled read list to be long exceeding the block size launched, either because of hardware constraints or user settings, the list would be partitioned to batches with the same size as the y dimensional size of the two dimensional block. At last, the results of each batch would be added up to acquire the final reuslt.

\chapter{Results}
Describe your results here.


\chapter{Discussion}
Discussion the significance or your results.


\section{Future Work}
Trying to investigate various possibilites of application for GPU to improve the performance, this paper provides the implementations of the general usage functions in CUDA. Unlike programs executed particularly on CPU, the design of GPU kernels often require extra attention, including memory copying schema and the detail implementation of algorithms. If we peurely consider datasets with shorter sequencing reads, the extent of parallelism could be further increased by computing different read sets concurrently. With a specific kernel for each case, warp divergence could be eliminated. Theoretically, this could benefit cases when the dataset consist of large amount of read sets, enormously reducing the amount of time to evaluate the likelihood by computing the results for each set all at once. Another possible optimization would be lessening the memory copying cost. Despite it not being the bottleneck of the program, some advanced techniques could be applied to lower the amount of data copied between the host CPU and the device even more. For example, if we are targeting only DNA and RNA sequences but not proteins, we could easily represent the bases, A, C, G, T(U for RNA), N, in 3 bits. Thus, instead of copying them as characters, they could be encoded first, enabling us to pack 2 bases into one unsigned integer. This is a common trick among sequencing related GPU applications, also adopted by other tools such as GASAL2 and NVBIO.

\chapter{Conclusion}
Add your conclusions here.


\newpage
\AddToContents{References}
\printbibliography


\end{document}
